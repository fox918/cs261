%Vorlage Uni, made by Oliver Wisler 25.20.2011

%set document language to german, standard font size and document size
\documentclass[ngerman, 12pt, pdftex]{scrartcl}[2006/07/30]

%encoding and input
\usepackage[ngerman]{babel} %spell correction
\usepackage[utf8]{inputenc} 
\usepackage[T1]{fontenc}
\usepackage[
	colorlinks=true,
	urlcolor=blue,
	linkcolor=green
]{hyperref}


%bugfixes
\usepackage{fixltx2e} 

%Font Symbols and Colors
\usepackage{textcomp} %more symbols
\usepackage{xcolor}

%Math
\usepackage{amsmath}
\usepackage{mathtools} %extends amsmath

%Programming
\usepackage{listingsutf8} %in utf8d
\lstset{language=Java,captionpos=b,tabsize=3,frame=lines,keywordstyle=\color{blue},commentstyle=\color{teal},stringstyle=\color{red},numbers=left,numberstyle=\tiny,numbersep=5pt,breaklines=true,showstringspaces=false,basicstyle=\footnotesize,emph={label},upquote=true} %Syntax highlighting

%Verbatim extension (with line numbers and tab-expansion)
\usepackage{moreverb} 

%Headers and Footer
\usepackage{fancyhdr}

%title
\title{CS261 Webprojekt}
\author{Frank Müller, Oliver Wisler}

\begin{document}
%declare  Header
\pagestyle{fancy}
\fancyhf{} 
\fancyhead[L]{cs261 Webprojekt} %left header
\fancyhead[C]{Dokumentation} %centered header
\fancyhead[R]{Frank Müller, Oliver Wisler}  % right header
\renewcommand{\headrulewidth}{0.1pt} 	%upper separating line
\fancyfoot[C]{\thepage} 				%centered footer, line number
%\renewcommand{\footrulewidth}{0.1pt} 	%lower separating line




%%%%%CONTENT%%%%%
%you might want to enable come features:
%\maketitle
\tableofcontents
\newpage


\section{Installation}

Für die Installation müssen folgende Schritte durchlaufen werden:
\subsection{Installation der Dateien}
    Bitte kopieren Sie alle Dateien in das gewünschte Stammverzeichnis.
    Ihr Webserver muss PHP und MySQL unterstützen, um diese Webseite anbieten zu können.
\subsection{Setzen der Konstanten}
    Setzten Sie alle wichtigen Konstanten in \verb+./config.php+.
    Die Verwendung und Bedeutung der Konstanten wird in der Datei erklärt.
    Achten Sie darauf, dass die Konstante \verb+INSTALL+ auf den Wert 1 gesetzt wird.
\subsection{Installation der Datenbank} % (fold)
\label{sub:Installation der Datenbank}
    Stellen Sie sicher, dass die in  \verb+./config.php+ angegebene Datenbank existiert.
    Führen Sie dann die Datei  \verb+./db_setup.php+ mit PHP aus. Alternativ können Sie auch 
    die URL  \url{http://www.IhreWebsite.ch/Pfad_zu_db_setup.php/db_setup.php} aufrufen.
% subsection Installation der Datenbank (end)
\subsection{Einrichten} % (fold)
\label{sub:Einrichten}
    Setzen Sie die Konstante \verb+INSTALL+ in \verb+config.php+ auf 0.
    Sie können sich nun im Webinterface mit folgenden Kenndaten anmelden: \\
    \begin{center}
    \begin{tabular}{ll}
        Benutzername & Passwort \\
        root & anderson \\
    \end{tabular}
    \end{center}


% subsection Einrichten (end)


% section  (end)

\section{Umsetzung}
\subsection{Funktionsumfang}


% subsection  (end)
\subsection{Aufbau}


% subsection  (end)
\subsection{Datenbank}


% subsection  (end)
% section  (end)

\section{Benutzeroberfläche}
\subsection{Verwaltung}

% subsection  (end)
\subsection{Lagerist}


% subsection  (end)

\subsection{Arbeiter}


% subsection  (end)
% section  (end)

\section{Fazit}
\subsection{Einschränkungen}

% subsection  (end)
\subsection{Problemstellen}

% subsection  (end)
\subsection{Lessons learned}

% subsection  (end)

% section  (end)
\end{document}]d
